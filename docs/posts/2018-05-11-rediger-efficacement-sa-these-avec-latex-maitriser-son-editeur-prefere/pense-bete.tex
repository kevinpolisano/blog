\documentclass[11pt]{amsart}
\usepackage[margin=1in]{geometry} 
\geometry{letterpaper}
\usepackage{graphicx}
\usepackage{amssymb}
\usepackage{amsmath}
\usepackage{epstopdf}
\DeclareGraphicsRule{.tif}{png}{.png}{`convert #1 `dirname #1`/`basename #1 .tif`.png}

\usepackage[francais]{babel}
\usepackage[utf8]{inputenc}
\usepackage[T1]{fontenc}

\usepackage{lscape}
\usepackage{longtable}
\usepackage{pdflscape}
\usepackage{tabularx,booktabs}
\usepackage{adjustbox}

\usepackage{array}
\newcolumntype{x}[1]{>{\centering\arraybackslash\hspace{0pt}}m{#1}}
\newcolumntype{y}[1]{>{\centering\arraybackslash\vspace{-0pt} }m{#1}}

\usepackage{fancybox}
\makeatletter
\newenvironment{CenteredBox}{% 
\begin{Sbox}}{% Save the content in a box
\end{Sbox}\centerline{\hspace{-2ex}\parbox[b][][b]{\wd\@Sbox}{\TheSbox}}}% And output it centered
\makeatother

\usepackage{color}
\usepackage{menukeys}
\usepackage{xcolor}
\definecolor{darkgreen}{rgb}{0,0.6,0}
\newcommand{\CodeSymbol}[1]{\textcolor{darkgreen}{#1}}

\usepackage{listings}
\lstset{
  language=[LaTeX]{TeX},
  basicstyle=\small,
  texcsstyle=*\bfseries\color{blue},
  keywordstyle=\color{red},
  literate={\{}{{\CodeSymbol{\{}}}1
  	       {\}}{{\CodeSymbol{\}}}}1
	       {]}{{\CodeSymbol{]}}}1
	       {[}{{\CodeSymbol{[}}}1
	       {\$}{{\CodeSymbol{\$}}}1,
  commentstyle=\color{red},
  keywords={$, \{, \}, [, ]},
  escapeinside={(*@}{@*)},          % if you want to add LaTeX within your code
  breaklines=true,
  }
  
\newcommand{\rep}{$\bullet$}

\title{Pense bête}
\author{Kévin Polisano}

\begin{document}
\maketitle

\begin{lstlisting}
\newcommand{\val}[1]{\left\lvert#1\right\rvert} 
\newcommand{\norme}[1]{\left\lVert#1\right\rVert}
\newcommand{\eu}{\ensuremath{\mathrm{e}}}
\newcommand{\ii}{\ensuremath{\mathrm{i}}}
\newcommand{\jj}{\ensuremath{\mathrm{j}}}
\newcommand{\expj}[1]{\ensuremath{\eu^{\,\jj#1}}}
\renewoperator{\Re}{\mathrm{Re}}{\nolimits}
\renewoperator{\Im}{\mathrm{Im}}{\nolimits}
\newcommand{\vect}[1]{\bm{#1}}
\newcommand{\mat}[1]{\mathbf{#1}}
\newcommand{\ope}[1]{\mathbf{#1}}
\newcommand{\ip}[2]{\left\langle#1,\,#2\right\rangle}
\end{lstlisting}

\begin{table}[htbp]
\caption{Raccourcis Mac traitement de texte}
\begin{tabular}{|c|c|}
\hline
Shortcut & Effet \\
\hline \hline
\cmd\,+\,\arrowkeyleft & Début de ligne\\ \hline
\cmd\,+\,\arrowkeyright & Fin de ligne\\ \hline
\cmd\,+\,\arrowkeyup & Début de document\\ \hline
\cmd\,+\,\arrowkeydown & Fin de document\\ \hline
\cmd\,+Z & Annuler l'opération\\ \hline
\cmd\,+A & Sélectionner tout\\ \hline
\cmd\,+F & Rechercher\\ \hline
\cmd\,+G & Rechercher à nouveau\\ \hline
\cmd\,+S & Enregistrer\\ \hline
\Alt\,+\, \arrowkeyleft & Se place au début du mot précédent\\ \hline
\Alt\,+\, \del & Supprimer le mot précédent\\ \hline
\Alt\,+\, \arrowkeyright & Se place à la fin du mot suivant\\ \hline
Fn+\,\del & Supprime le caractère à droite \\ \hline
\cmd\,+\,\shift\,+\arrowkeyleft & Sélectionne à gauche jusqu'en début de ligne \\ \hline
\cmd\,+\,\shift\,+\arrowkeyright & Sélectionne à droite jusqu'en fin de ligne \\ \hline
$\wedge$\,+\,K & Supprimer jusqu'à la fin de ligne \\ \hline
\shift\,+\,\arrowkeyup & \'Etendre la sélection à la ligne du dessus\\ \hline
\shift\,+\,\arrowkeydown & \'Etendre la sélection à la ligne du dessous\\ \hline
\shift\,+\,\arrowkeyleft & \'Etendre la sélection d'un caractère à gauche\\ \hline
\shift\,+\,\arrowkeyright & \'Etendre la sélection d'un caractère à droite\\ \hline
\shift\,+\,\Alt\,+\,\arrowkeyleft & \'Etendre la sélection d'un mot à gauche\\ \hline
\shift\,+\,\Alt\,+\,\arrowkeyright & \'Etendre la sélection d'un mot à droite\\ \hline
$\wedge$\,+L & Center l'éditeur sur la position du curseur\\ \hline
\end{tabular}
\end{table}

\begin{table}[t]
\caption{Macros TeXShop}
\begin{tabular}{|c|c|}
\hline
Macro & Résultat \\
\hline \hline
\cmd\,+: & $\backslash$ \\ \hline
\cmd\,+( & [ \\ \hline
\cmd\,+) & ] \\ \hline
\cmd\,+L & 
\begin{lstlisting}
\val{(*@ \rep @*)}
\end{lstlisting} \\ \hline
\cmd\,+\Alt\,+L & 
\begin{lstlisting}
\norme{(*@ \rep @*)}
\end{lstlisting} \\ \hline
\cmd\,+B & 
\begin{lstlisting}
\vect{(*@ \rep @*)}
\end{lstlisting} \\ \hline
\cmd\,+\Alt\,+B & 
\begin{lstlisting}
\mat{(*@ \rep @*)}
\end{lstlisting} \\ \hline
\cmd\,+\Alt\,+$\wedge$\,+B & 
\begin{lstlisting}
\ope{(*@ \rep @*)}
\end{lstlisting} \\ \hline
\cmd\,+P & 
\begin{lstlisting}
^{(*@ \rep @*)}
\end{lstlisting} \\ \hline
\cmd\,+- & 
\begin{lstlisting}
_{(*@ \rep @*)}
\end{lstlisting} \\ \hline
\cmd\,+< & 
\begin{lstlisting}
\ip{(*@ \rep @*)}{(*@ \rep @*)}
\end{lstlisting} \\ \hline
\cmd\,+\Alt\,+( & 
\begin{lstlisting}
\left((*@ \rep @*)\right)
\end{lstlisting} \\ \hline
\cmd\,+\Alt\,+$\wedge$\,+( & 
\begin{lstlisting}
\left[(*@ \rep @*)\right]
\end{lstlisting} \\ \hline
\cmd\,+\shift\,+\Alt\,+$\wedge$\,+( & 
\begin{lstlisting}
\left\{(*@ \rep @*)\right\}
\end{lstlisting} \\ \hline

\end{tabular}
\end{table}

\begin{table}[t]
\caption{Raccourcis claviers TeXShop}
\begin{tabular}{|c|c|}
\hline
Shortcut & Résultat \\
\hline \hline
\Alt\,+/ & 
\begin{lstlisting}
\frac{(*@ \rep @*)}{(*@ \rep @*)}
\end{lstlisting} \\ \hline
\Alt\,+< & \lstinline{\leq} \\ \hline
\Alt\,+\shift\,+< & \lstinline{\geq} \\ \hline
\Alt\,+p & \lstinline{\pi} \\ \hline
\Alt\,+, & \lstinline{\infty} \\ \hline
\end{tabular}
\end{table}

%%%%%%%%%%%%%%%%%%%%% ENVIRONNEMENTS THEOREM %%%%%%%%%%%%%%%%%%%%%%%%%%%%%%%%%%%%%%%%
\begin{landscape}
\small  % Switch from 12pt to 11pt; otherwise, table won't fit
\setlength\LTleft{0pt}            % default: \parindent
\setlength\LTright{0pt}           % default: \fill
\begin{table}[t]
\caption{Environnements Theorem}
\begin{adjustbox}{width=\linewidth}
\begin{tabular}{|x{2.6cm}|y{15cm}|x{5cm}|}
\hline 
Commande complétion &    
 Macro correspondante & Résultat produit \\
\hline \hline
% Environnement Theorem ----------------------------------------------------------------------------------------------------------------------------------------------------------------
btheo
&
\begin{CenteredBox}
  \begin{lstlisting}
btheo:=\begin{theorem}#RET##INS#(*@ \rep @*)#INS##RET#\end{theorem}#RET#(*@ \rep @*)
\end{lstlisting}
\end{CenteredBox}
& 
\begin{CenteredBox}
  \begin{lstlisting}
  \begin{theorem}
  (*@ \rep @*)
  \end{theorem}
  (*@ \rep @*)
  \end{lstlisting}
\end{CenteredBox}
  \\ \hline
 % Environnement Proposition ----------------------------------------------------------------------------------------------------------------------------------------------------------------
bprop
&
\begin{CenteredBox}
  \begin{lstlisting}
bprop:=\begin{proposition}#RET##INS#(*@ \rep @*)#INS##RET#\end{proposition}#RET#(*@ \rep @*)
\end{lstlisting}
\end{CenteredBox}
& 
\begin{CenteredBox}
  \begin{lstlisting}
  \begin{proposition}
  (*@ \rep @*)
  \end{proposition}
  (*@ \rep @*)
  \end{lstlisting}
\end{CenteredBox}
  \\ \hline
 % Environnement Definition ----------------------------------------------------------------------------------------------------------------------------------------------------------------
bdef
&
\begin{CenteredBox}
  \begin{lstlisting}
bdef:=\begin{definition}#RET##INS#(*@ \rep @*)#INS##RET#\end{definition}#RET#(*@ \rep @*)
\end{lstlisting}
\end{CenteredBox}
& 
\begin{CenteredBox}
  \begin{lstlisting}
  \begin{definition}
  (*@ \rep @*)
  \end{definition}
  (*@ \rep @*)
  \end{lstlisting}
\end{CenteredBox}
  \\ \hline
   % Environnement Corollary ----------------------------------------------------------------------------------------------------------------------------------------------------------------
bcor
&
\begin{CenteredBox}
  \begin{lstlisting}
bcor:=\begin{corollary}#RET##INS#(*@ \rep @*)#INS##RET#\end{corollary}#RET#(*@ \rep @*)
\end{lstlisting}
\end{CenteredBox}
& 
\begin{CenteredBox}
  \begin{lstlisting}
  \begin{corollary}
  (*@ \rep @*)
  \end{corollary}
  (*@ \rep @*)
  \end{lstlisting}
\end{CenteredBox}
  \\ \hline
   % Environnement Lemma ----------------------------------------------------------------------------------------------------------------------------------------------------------------
blem
&
\begin{CenteredBox}
  \begin{lstlisting}
blem:=\begin{lemma}#RET##INS#(*@ \rep @*)#INS##RET#\end{lemma}#RET#(*@ \rep @*)
\end{lstlisting}
\end{CenteredBox}
& 
\begin{CenteredBox}
  \begin{lstlisting}
  \begin{lemma}
  (*@ \rep @*)
  \end{lemma}
  (*@ \rep @*)
  \end{lstlisting}
\end{CenteredBox}
  \\ \hline
 % Environnement Proof ----------------------------------------------------------------------------------------------------------------------------------------------------------------
bproof
&
\begin{CenteredBox}
  \begin{lstlisting}
bproof:=\begin{proof}#RET##INS#(*@ \rep @*)#INS##RET#\end{proof}#RET#(*@ \rep @*)
\end{lstlisting}
\end{CenteredBox}
& 
\begin{CenteredBox}
  \begin{lstlisting}
  \begin{proof}
  (*@ \rep @*)
  \end{proof}
  (*@ \rep @*)
  \end{lstlisting}
\end{CenteredBox}
  \\ \hline
\end{tabular}
\end{adjustbox}
\end{table}
\end{landscape}

%%%%%%%%%%%%%%%%%%%%% ENVIRONNEMENTS TABLEAUX %%%%%%%%%%%%%%%%%%%%%%%%%%%%%%%%%%%%%%%%
\begin{landscape}
\small  % Switch from 12pt to 11pt; otherwise, table won't fit
\setlength\LTleft{0pt}            % default: \parindent
\setlength\LTright{0pt}           % default: \fill
\begin{table}[t]
\caption{Environnements tableaux}
\begin{adjustbox}{width=\linewidth}
\begin{tabular}{|x{2.6cm}|y{16cm}|x{6cm}|}
\hline 
Commande complétion &    
 Macro correspondante & Résultat produit \\
\hline \hline
 % Environnement Tabular ----------------------------------------------------------------------------------------------------------------------------------------------------------------
btab
&
\begin{CenteredBox}
  \begin{lstlisting}
btab:=\begin{tabular}{#INS#(*@ \rep @*)#INS#}#RET#(*@ \rep @*)#RET#\end{tabular}#RET#(*@ \rep @*)
\end{lstlisting}
\end{CenteredBox}
& 
\begin{CenteredBox}
  \begin{lstlisting}
  \begin{tabular}{(*@ \rep @*)}
  (*@ \rep @*)
  \end{tabular}
  (*@ \rep @*)
  \end{lstlisting}
\end{CenteredBox}
  \\ \hline
   % Environnement Tabularx ----------------------------------------------------------------------------------------------------------------------------------------------------------------
btabx
&
\begin{CenteredBox}
  \begin{lstlisting}
btabx:=\begin{tabularx}{#INS#(*@ \rep @*)#INS#}{(*@ \rep @*)}#RET#(*@ \rep @*)#RET#\end{tabularx}#RET#(*@ \rep @*)
\end{lstlisting}
\end{CenteredBox}
& 
\begin{CenteredBox}
  \begin{lstlisting}
  \begin{tabularx}{(*@ \rep @*)}{(*@ \rep @*)}
  (*@ \rep @*)
  \end{tabularx}
  (*@ \rep @*)
  \end{lstlisting}
\end{CenteredBox}
  \\ \hline
   % Environnement Table ----------------------------------------------------------------------------------------------------------------------------------------------------------------
btabl
&
\begin{CenteredBox}
  \begin{lstlisting}
btabl:=\begin{table}{#INS#(*@ \rep @*)#INS#}#RET#(*@ \rep @*)#RET#\end{table}#RET#(*@ \rep @*)
\end{lstlisting}
\end{CenteredBox}
& 
\begin{CenteredBox}
  \begin{lstlisting}
  \begin{table}
  (*@ \rep @*)
  \end{table}
  (*@ \rep @*)
  \end{lstlisting}
\end{CenteredBox}
  \\ \hline
  % Environnement Array ----------------------------------------------------------------------------------------------------------------------------------------------------------------
barr
&
\begin{CenteredBox}
  \begin{lstlisting}
barr:=\begin{array}#RET##INS#(*@ \rep @*)#INS##RET#\end{array}(*@ \rep @*)
\end{lstlisting}
\end{CenteredBox}
& 
\begin{CenteredBox}
  \begin{lstlisting}
  \begin{array}
  (*@ \rep @*)
  \end{array}
  (*@ \rep @*)
  \end{lstlisting}
\end{CenteredBox}
  \\ \hline
 % Environnement (p)Matrix ----------------------------------------------------------------------------------------------------------------------------------------------------------------
bpmat
&
\begin{CenteredBox}
  \begin{lstlisting}
bpmat:=\begin{pmatrix}#RET##INS#(*@ \rep @*)#INS##RET#\end{pmatrix}#RET#(*@ \rep @*)
\end{lstlisting}
\end{CenteredBox}
& 
\begin{CenteredBox}
  \begin{lstlisting}
  \begin{pmatrix}
  (*@ \rep @*)
  \end{pmatrix}
  (*@ \rep @*)
  \end{lstlisting}
\end{CenteredBox}
  \\ \hline
  % Environnement (b)Matrix ----------------------------------------------------------------------------------------------------------------------------------------------------------------
bbmat
&
\begin{CenteredBox}
  \begin{lstlisting}
bbmat:=\begin{bmatrix}#RET##INS#(*@ \rep @*)#INS##RET#\end{bmatrix}#RET#(*@ \rep @*)
\end{lstlisting}
\end{CenteredBox}
& 
\begin{CenteredBox}
  \begin{lstlisting}
  \begin{bmatrix}
  (*@ \rep @*)
  \end{bmatrix}
  (*@ \rep @*)
  \end{lstlisting}
\end{CenteredBox}
  \\ \hline 
 % Environnement (v)Matrix ----------------------------------------------------------------------------------------------------------------------------------------------------------------
bvmat
&
\begin{CenteredBox}
  \begin{lstlisting}
bvmat:=\begin{vmatrix}#RET##INS#(*@ \rep @*)#INS##RET#\end{vmatrix}#RET#(*@ \rep @*)
\end{lstlisting}
\end{CenteredBox}
& 
\begin{CenteredBox}
  \begin{lstlisting}
  \begin{vmatrix}
  (*@ \rep @*)
  \end{vmatrix}
  (*@ \rep @*)
  \end{lstlisting}
\end{CenteredBox}
  \\ \hline
\end{tabular}
\end{adjustbox}
\end{table}
\end{landscape}

%%%%%%%%%%%%%%%%%%%%% ENVIRONNEMENTS EQUATION %%%%%%%%%%%%%%%%%%%%%%%%%%%%%%%%%%%%%%%%
\begin{landscape}
\small  % Switch from 12pt to 11pt; otherwise, table won't fit
\setlength\LTleft{0pt}            % default: \parindent
\setlength\LTright{0pt}           % default: \fill
\begin{table}[t]
\caption{Environnements équation}
\begin{adjustbox}{width=\linewidth}
\begin{tabular}{|x{2.6cm}|y{15cm}|x{5cm}|}
\hline 
Commande complétion &    
 Macro correspondante & Résultat produit \\
\hline \hline
 % Environnement Equation ----------------------------------------------------------------------------------------------------------------------------------------------------------------
beq 
&
\vspace{2pt}
\begin{CenteredBox}
  \begin{lstlisting}
beq:=\begin{equation}#RET##INS#(*@ \rep @*)#INS##RET#\end{equation}#RET#(*@ \rep @*)
\end{lstlisting}
\end{CenteredBox}
& 
\begin{CenteredBox}
  \begin{lstlisting}
  \begin{equation}
  (*@ \rep @*)
  \end{equation}
  (*@ \rep @*)
  \end{lstlisting}
\end{CenteredBox}
 \\ \hline
 % Environnement Align ----------------------------------------------------------------------------------------------------------------------------------------------------------------
bali
&
\vspace{2pt}
\begin{CenteredBox}
  \begin{lstlisting}
bali:=\begin{align}#RET##INS#(*@ \rep @*)#INS##RET#\end{align}#RET#(*@ \rep @*)
\end{lstlisting}
\end{CenteredBox}
& 
\begin{CenteredBox}
  \begin{lstlisting}
  \begin{align}
  (*@ \rep @*)
  \end{align}
  (*@ \rep @*)
  \end{lstlisting}
\end{CenteredBox}
 \\ \hline
  % Environnement Case ----------------------------------------------------------------------------------------------------------------------------------------------------------------
bcase
&
\vspace{2pt}
\begin{CenteredBox}
  \begin{lstlisting}
bcase:=\begin{cases}#RET##INS#(*@ \rep @*)#INS##RET#\end{cases}#RET#(*@ \rep @*)
\end{lstlisting}
\end{CenteredBox}
& 
\begin{CenteredBox}
  \begin{lstlisting}
  \begin{cases}
  (*@ \rep @*)
  \end{cases}
  (*@ \rep @*)
  \end{lstlisting}
\end{CenteredBox}
 \\ \hline
   % Environnement Center ----------------------------------------------------------------------------------------------------------------------------------------------------------------
bcent
&
\vspace{2pt}
\begin{CenteredBox}
  \begin{lstlisting}
bcent:=\begin{center}#RET##INS#(*@ \rep @*)#INS##RET#\end{center}#RET#(*@ \rep @*)
\end{lstlisting}
\end{CenteredBox}
& 
\begin{CenteredBox}
  \begin{lstlisting}
  \begin{center}
  (*@ \rep @*)
  \end{center}
  (*@ \rep @*)
  \end{lstlisting}
\end{CenteredBox}
 \\ \hline
   % Environnement Verbatim ----------------------------------------------------------------------------------------------------------------------------------------------------------------
bverb
&
\vspace{2pt}
\begin{CenteredBox}
  \begin{lstlisting}
bverb:=\begin{verbatim}#RET##INS#(*@ \rep @*)#INS##RET#\end{verbatim}#RET#(*@ \rep @*)
\end{lstlisting}
\end{CenteredBox}
& 
\begin{CenteredBox}
  \begin{lstlisting}
  \begin{verbatim}
  (*@ \rep @*)
  \end{verbatim}
  (*@ \rep @*)
  \end{lstlisting}
\end{CenteredBox}
 \\ \hline
  % Environnement Enumerate ----------------------------------------------------------------------------------------------------------------------------------------------------------------
benu
&
\vspace{2pt}
\begin{CenteredBox}
  \begin{lstlisting}
benu:=\begin{enumerate}#RET#\item#INS#(*@ \rep @*)#INS##RET#\end{enumerate}#RET#(*@ \rep @*)
\end{lstlisting}
\end{CenteredBox}
& 
\begin{CenteredBox}
  \begin{lstlisting}
  \begin{enumerate}
  \item (*@ \rep @*)
  \end{enumerate}
  (*@ \rep @*)
  \end{lstlisting}
\end{CenteredBox}
 \\ \hline
  % Environnement Itemize ----------------------------------------------------------------------------------------------------------------------------------------------------------------
bite
&
\vspace{2pt}
\begin{CenteredBox}
  \begin{lstlisting}
bite:=\begin{itemize}#RET#\item#INS#(*@ \rep @*)#INS##RET#\end{itemize}#RET#(*@ \rep @*)
\end{lstlisting}
\end{CenteredBox}
& 
\begin{CenteredBox}
  \begin{lstlisting}
  \begin{itemize}
  \item (*@ \rep @*)
  \end{itemize}
  (*@ \rep @*)
  \end{lstlisting}
\end{CenteredBox}
 \\ \hline

  \end{tabular}
\end{adjustbox}
\end{table}
\end{landscape}

%%%%%%%%%%%%%%%%%%%%% ENVIRONNEMENTS GRAPHIQUE %%%%%%%%%%%%%%%%%%%%%%%%%%%%%%%%%%%%%%%%
\begin{landscape}
\small  % Switch from 12pt to 11pt; otherwise, table won't fit
\setlength\LTleft{0pt}            % default: \parindent
\setlength\LTright{0pt}           % default: \fill
\begin{table}[t]
\caption{Environnements et commandes graphiques}
\begin{adjustbox}{width=\linewidth}
\begin{tabular}{|x{2.6cm}|y{15cm}|x{5cm}|}
\hline 
Commande complétion &    
 Macro correspondante & Résultat produit \\
\hline \hline
 % Environnement Figure ----------------------------------------------------------------------------------------------------------------------------------------------------------------
bite
&
\vspace{2pt}
\begin{CenteredBox}
  \begin{lstlisting}
bfig:=\begin{figure}#RET##INS#(*@ \rep @*)#INS##RET#\end{figure}#RET#(*@ \rep @*)
\end{lstlisting}
\end{CenteredBox}
& 
\begin{CenteredBox}
  \begin{lstlisting}
  \begin{figure}
  (*@ \rep @*)
  \end{figure}
  (*@ \rep @*)
  \end{lstlisting}
\end{CenteredBox}
 \\ \hline
% Commande centering ----------------------------------------------------------------------------------------------------------------------------------------------------------------
center
&
\vspace{2pt}
\begin{CenteredBox}
  \begin{lstlisting}
center:=centering
\end{lstlisting}
\end{CenteredBox}
& 
\begin{CenteredBox}
  \begin{lstlisting}
  \centering
  \end{lstlisting}
\end{CenteredBox}
 \\ \hline
 % Commande caption ----------------------------------------------------------------------------------------------------------------------------------------------------------------
cap
&
\vspace{2pt}
\begin{CenteredBox}
  \begin{lstlisting}
cap:=\caption{#INS#(*@ \rep @*)#INS#}
\end{lstlisting}
\end{CenteredBox}
& 
\begin{CenteredBox}
  \begin{lstlisting}
  \caption{(*@ \rep @*)}
  \end{lstlisting}
\end{CenteredBox}
 \\ \hline
 % Commande includegraphics ----------------------------------------------------------------------------------------------------------------------------------------------------------------
incg
&
\vspace{2pt}
\begin{CenteredBox}
  \begin{lstlisting}
incg:=\includegraphics{#INS#(*@ \rep @*)#INS#}#RET#
\end{lstlisting}
\end{CenteredBox}
& 
\begin{CenteredBox}
  \begin{lstlisting}
  \includegraphics{(*@ \rep @*)}
  \end{lstlisting}
\end{CenteredBox}
 \\ \hline
% Commande includegraphics tabular ----------------------------------------------------------------------------------------------------------------------------------------------------------------
incgt
&
\vspace{2pt}
\begin{CenteredBox}
  \begin{lstlisting}
incgt:=\begin{figure}#RET#\centering#RET#\begin{tabular}{cc}#RET#
\includegraphics{#INS#(*@ \rep @*)#INS#}&\includegraphics{(*@ \rep @*)}
#RET#\end{tabular}#RET#\caption{(*@ \rep @*)}#RET#\end{figure}#RET#(*@ \rep @*)
\end{lstlisting}
\end{CenteredBox}
& 
\begin{CenteredBox}
  \begin{lstlisting}
  \begin{figure}
  \centering
  \begin{tabular}{cc}
  \includegraphics{(*@ \rep @*)}&
  \includegraphics{(*@ \rep @*)}
  \end{tabular}
  \caption{(*@ \rep @*)}
  \end{figure}
  (*@ \rep @*)
  \end{lstlisting}
\end{CenteredBox}
 \\ \hline
 % Environnement Tikzpicture ----------------------------------------------------------------------------------------------------------------------------------------------------------------
btikz
&
\vspace{2pt}
\begin{CenteredBox}
  \begin{lstlisting}
btikz:=\begin{tikzpicture}#RET##INS#(*@ \rep @*)#INS##RET#\end{tikzpicture}#RET#(*@ \rep @*)
\end{lstlisting}
\end{CenteredBox}
& 
\begin{CenteredBox}
  \begin{lstlisting}
  \begin{tikzpicture}
  (*@ \rep @*)
  \end{tikzpicture}
  (*@ \rep @*)
  \end{lstlisting}
\end{CenteredBox}
 \\ \hline

  \end{tabular}
\end{adjustbox}
\end{table}
\end{landscape}

%%%%%%%%%%%%%%%%%%%%% ORGANISATION DOCUMENT %%%%%%%%%%%%%%%%%%%%%%%%%%%%%%%%%%%%%%%%
\begin{landscape}
\small  % Switch from 12pt to 11pt; otherwise, table won't fit
\setlength\LTleft{0pt}            % default: \parindent
\setlength\LTright{0pt}           % default: \fill
\begin{table}[t]
\caption{Organisation du document}
\begin{adjustbox}{width=\linewidth}
\begin{tabular}{|x{2.6cm}|y{15cm}|x{5cm}|}
\hline 
Commande complétion &    
 Macro correspondante & Résultat produit \\
\hline \hline
% Environnement Abstract ----------------------------------------------------------------------------------------------------------------------------------------------------------------
babs
&
\vspace{2pt}
\begin{CenteredBox}
  \begin{lstlisting}
babs:=\begin{abstract}#RET##INS#(*@ \rep @*)#INS##RET#\end{abstract}#RET#(*@ \rep @*)
\end{lstlisting}
\end{CenteredBox}
& 
\begin{CenteredBox}
  \begin{lstlisting}
  \begin{abstract}
  (*@ \rep @*)
  \end{abstract}
  (*@ \rep @*)
  \end{lstlisting}
\end{CenteredBox}
 \\ \hline
 % Environnement Appendix ----------------------------------------------------------------------------------------------------------------------------------------------------------------
bapp
&
\vspace{2pt}
\begin{CenteredBox}
  \begin{lstlisting}
bapp:=\begin{appendix}#RET##INS#(*@ \rep @*)#INS##RET#\end{appendix}#RET#(*@ \rep @*)
\end{lstlisting}
\end{CenteredBox}
& 
\begin{CenteredBox}
  \begin{lstlisting}
  \begin{appendix}
  (*@ \rep @*)
  \end{appendix}
  (*@ \rep @*)
  \end{lstlisting}
\end{CenteredBox}
 \\ \hline
 % Commande chapter ----------------------------------------------------------------------------------------------------------------------------------------------------------------
chap
&
\vspace{2pt}
\begin{CenteredBox}
  \begin{lstlisting}
chap:=\chapter{#INS#(*@ \rep @*)#INS#}#RET#(*@ \rep @*)
\end{lstlisting}
\end{CenteredBox}
& 
\begin{CenteredBox}
  \begin{lstlisting}
  \chapter{(*@ \rep @*)}
  (*@ \rep @*)
  \end{lstlisting}
\end{CenteredBox}
 \\ \hline
  % Commande section ----------------------------------------------------------------------------------------------------------------------------------------------------------------
sec
&
\vspace{2pt}
\begin{CenteredBox}
  \begin{lstlisting}
sec:=\section{#INS#(*@ \rep @*)#INS#}#RET#(*@ \rep @*)
\end{lstlisting}
\end{CenteredBox}
& 
\begin{CenteredBox}
  \begin{lstlisting}
  \section{(*@ \rep @*)}
  (*@ \rep @*)
  \end{lstlisting}
\end{CenteredBox}
 \\ \hline
   % Commande subsection ----------------------------------------------------------------------------------------------------------------------------------------------------------------
ssec
&
\vspace{2pt}
\begin{CenteredBox}
  \begin{lstlisting}
ssec:=\subsection{#INS#(*@ \rep @*)#INS#}#RET#(*@ \rep @*)
\end{lstlisting}
\end{CenteredBox}
& 
\begin{CenteredBox}
  \begin{lstlisting}
  \subsection{(*@ \rep @*)}
  (*@ \rep @*)
  \end{lstlisting}
\end{CenteredBox}
 \\ \hline
   % Commande subsubsection ----------------------------------------------------------------------------------------------------------------------------------------------------------------
sssec
&
\vspace{2pt}
\begin{CenteredBox}
  \begin{lstlisting}
sssec:=\subsubsection{#INS#(*@ \rep @*)#INS#}#RET#(*@ \rep @*)
\end{lstlisting}
\end{CenteredBox}
& 
\begin{CenteredBox}
  \begin{lstlisting}
  \subsubsection{(*@ \rep @*)}
  (*@ \rep @*)
  \end{lstlisting}
\end{CenteredBox}
 \\ \hline
    % Commande bibliography / bibliographystyle ----------------------------------------------------------------------------------------------------------------------------------------------------------------
bib
&
\vspace{2pt}
\begin{CenteredBox}
  \begin{lstlisting}
bib:=\bibliography{#INS#(*@ \rep @*)#INS#}#RET#(*@ \rep @*)
\end{lstlisting}
\end{CenteredBox}
& 
\begin{CenteredBox}
  \begin{lstlisting}
  \bibliography{(*@ \rep @*)}
  (*@ \rep @*)
  \end{lstlisting}
\end{CenteredBox}
 \\ \hline

   \end{tabular}
\end{adjustbox}
\end{table}
\end{landscape}
 
 %%%%%%%%%%%%%%%%%%%%% COMMADES DOCUMENT %%%%%%%%%%%%%%%%%%%%%%%%%%%%%%%%%%%%%%%%
\begin{landscape}
\small  % Switch from 12pt to 11pt; otherwise, table won't fit
\setlength\LTleft{0pt}            % default: \parindent
\setlength\LTright{0pt}           % default: \fill
\begin{table}[t]
\caption{Commandes du document}
\begin{adjustbox}{width=\linewidth}
\begin{tabular}{|x{2.6cm}|y{15cm}|x{5cm}|}
\hline 
Commande complétion &    
 Macro correspondante & Résultat produit \\
\hline \hline

  % Commande newcommand ----------------------------------------------------------------------------------------------------------------------------------------------------------------
ncm
&
\vspace{2pt}
\begin{CenteredBox}
  \begin{lstlisting}
ncm:=\newcommand{#INS#(*@ \rep @*)#INS#}{(*@ \rep @*)}#RET#(*@ \rep @*)
\end{lstlisting}
\end{CenteredBox}
& 
\begin{CenteredBox}
  \begin{lstlisting}
  \newcommand{(*@ \rep @*)}{(*@ \rep @*)}
  (*@ \rep @*)
  \end{lstlisting}
\end{CenteredBox}
 \\ \hline
   % Commande renewcommand ----------------------------------------------------------------------------------------------------------------------------------------------------------------
rncm
&
\vspace{2pt}
\begin{CenteredBox}
  \begin{lstlisting}
ncm:=\renewcommand{#INS#(*@ \rep @*)#INS#}{(*@ \rep @*)}#RET#(*@ \rep @*)
\end{lstlisting}
\end{CenteredBox}
& 
\begin{CenteredBox}
  \begin{lstlisting}
  \renewcommand{(*@ \rep @*)}{(*@ \rep @*)}
  (*@ \rep @*)
  \end{lstlisting}
\end{CenteredBox}
 \\ \hline
   % Commande newpage ----------------------------------------------------------------------------------------------------------------------------------------------------------------
npg
&
\vspace{2pt}
\begin{CenteredBox}
  \begin{lstlisting}
npg:=\newpage #RET#
\end{lstlisting}
\end{CenteredBox}
& 
\begin{CenteredBox}
  \begin{lstlisting}
  \newpage
  \end{lstlisting}
\end{CenteredBox}
 \\ \hline
   % Commande newenvironnement ----------------------------------------------------------------------------------------------------------------------------------------------------------------
nenv
&
\vspace{2pt}
\begin{CenteredBox}
  \begin{lstlisting}
nenv:=\newenvironnement{#INS#(*@ \rep @*)#INS#}{(*@ \rep @*)}{(*@ \rep @*)}#RET#(*@ \rep @*)
\end{lstlisting}
\end{CenteredBox}
& 
\begin{CenteredBox}
  \begin{lstlisting}
  \newcommand{(*@ \rep @*)}{(*@ \rep @*)}{(*@ \rep @*)}
  (*@ \rep @*)
  \end{lstlisting}
\end{CenteredBox}
 \\ \hline
    % Commande label ----------------------------------------------------------------------------------------------------------------------------------------------------------------
lbl
&
\vspace{2pt}
\begin{CenteredBox}
  \begin{lstlisting}
lbl:=\label{#INS#(*@ \rep @*)#INS#}(*@ \rep @*)
\end{lstlisting}
\end{CenteredBox}
& 
\begin{CenteredBox}
  \begin{lstlisting}
  \label{(*@ \rep @*)}(*@ \rep @*)
  \end{lstlisting}
\end{CenteredBox}
 \\ \hline
  % Commande ref ----------------------------------------------------------------------------------------------------------------------------------------------------------------
ref
&
\vspace{2pt}
\begin{CenteredBox}
  \begin{lstlisting}
ref:=\ref{#INS#(*@ \rep @*)#INS#}(*@ \rep @*)
\end{lstlisting}
\end{CenteredBox}
& 
\begin{CenteredBox}
  \begin{lstlisting}
  \ref{(*@ \rep @*)}(*@ \rep @*)
  \end{lstlisting}
\end{CenteredBox}
 \\ \hline
   % Commande href ----------------------------------------------------------------------------------------------------------------------------------------------------------------
href
&
\vspace{2pt}
\begin{CenteredBox}
  \begin{lstlisting}
href:=\ref{#INS#(*@ \rep @*)#INS#}{(*@ \rep @*)}(*@ \rep @*)
\end{lstlisting}
\end{CenteredBox}
& 
\begin{CenteredBox}
  \begin{lstlisting}
  \href{(*@ \rep @*)}{(*@ \rep @*)}(*@ \rep @*)
  \end{lstlisting}
\end{CenteredBox}
 \\ \hline
   % Commande url ----------------------------------------------------------------------------------------------------------------------------------------------------------------
url
&
\vspace{2pt}
\begin{CenteredBox}
  \begin{lstlisting}
url:=\url{#INS#(*@ \rep @*)#INS#}(*@ \rep @*)
\end{lstlisting}
\end{CenteredBox}
& 
\begin{CenteredBox}
  \begin{lstlisting}
  \url{(*@ \rep @*)}(*@ \rep @*)
  \end{lstlisting}
\end{CenteredBox}
 \\ \hline
   % Commande cite ----------------------------------------------------------------------------------------------------------------------------------------------------------------
ci
&
\vspace{2pt}
\begin{CenteredBox}
  \begin{lstlisting}
ci:=\cite{#INS#(*@ \rep @*)#INS#}(*@ \rep @*)
\end{lstlisting}
\end{CenteredBox}
& 
\begin{CenteredBox}
  \begin{lstlisting}
  \cite{(*@ \rep @*)}(*@ \rep @*)
  \end{lstlisting}
\end{CenteredBox}
 \\ \hline
    % Commande usepackage ----------------------------------------------------------------------------------------------------------------------------------------------------------------
usep
&
\vspace{2pt}
\begin{CenteredBox}
  \begin{lstlisting}
usep:=\usepackage{#INS#(*@ \rep @*)#INS#}{(*@ \rep @*)}(*@ \rep @*)
\end{lstlisting}
\end{CenteredBox}
& 
\begin{CenteredBox}
  \begin{lstlisting}
  \usepackage{(*@ \rep @*)}{(*@ \rep @*)}(*@ \rep @*)
  \end{lstlisting}
\end{CenteredBox}
 \\ \hline
 
  \end{tabular}
\end{adjustbox}
\end{table}
\end{landscape}

%%%%%%%%%%%%%%%%%%%%% MISE EN FORME TEXTE %%%%%%%%%%%%%%%%%%%%%%%%%%%%%%%%%%%%%%%%
\begin{landscape}
\small  % Switch from 12pt to 11pt; otherwise, table won't fit
\setlength\LTleft{0pt}            % default: \parindent
\setlength\LTright{0pt}           % default: \fill
\begin{table}[t]
\caption{Mise en forme du texte}
\begin{adjustbox}{width=\linewidth}
\begin{tabular}{|x{2.6cm}|y{15cm}|x{5cm}|}
\hline 
Commande complétion &    
 Macro correspondante & Résultat produit \\
\hline \hline
  % Commande textit ----------------------------------------------------------------------------------------------------------------------------------------------------------------
it
&
\vspace{2pt}
\begin{CenteredBox}
  \begin{lstlisting}
it:=\textit{#INS#(*@ \rep @*)#INS#}(*@ \rep @*)
\end{lstlisting}
\end{CenteredBox}
& 
\textit{texte}
 \\ \hline
  % Commande emph ----------------------------------------------------------------------------------------------------------------------------------------------------------------
emph
&
\vspace{2pt}
\begin{CenteredBox}
  \begin{lstlisting}
emph:=\emph{#INS#(*@ \rep @*)#INS#}(*@ \rep @*)
\end{lstlisting}
\end{CenteredBox}
& 
\emph{texte}
 \\ \hline
   % Commande textbf ----------------------------------------------------------------------------------------------------------------------------------------------------------------
bf
&
\vspace{2pt}
\begin{CenteredBox}
  \begin{lstlisting}
bf:=\textbf{#INS#(*@ \rep @*)#INS#}(*@ \rep @*)
\end{lstlisting}
\end{CenteredBox}
& 
\textbf{texte}
 \\ \hline
  % Commande textsf ----------------------------------------------------------------------------------------------------------------------------------------------------------------
sf
&
\vspace{2pt}
\begin{CenteredBox}
  \begin{lstlisting}
sf:=\textsf{#INS#(*@ \rep @*)#INS#}(*@ \rep @*)
\end{lstlisting}
\end{CenteredBox}
& 
\textsf{texte}
 \\ \hline
  % Commande textsc ----------------------------------------------------------------------------------------------------------------------------------------------------------------
sc
&
\vspace{2pt}
\begin{CenteredBox}
  \begin{lstlisting}
sc:=\textsc{#INS#(*@ \rep @*)#INS#}(*@ \rep @*)
\end{lstlisting}
\end{CenteredBox}
& 
\textsc{texte}
 \\ \hline
  % Commande textbf ----------------------------------------------------------------------------------------------------------------------------------------------------------------
sl
&
\vspace{2pt}
\begin{CenteredBox}
  \begin{lstlisting}
sl:=\textsl{#INS#(*@ \rep @*)#INS#}(*@ \rep @*)
\end{lstlisting}
\end{CenteredBox}
& 
\textsl{texte}
 \\ \hline
  % Commande textbf ----------------------------------------------------------------------------------------------------------------------------------------------------------------
tt
&
\vspace{2pt}
\begin{CenteredBox}
  \begin{lstlisting}
tt:=\texttt{#INS#(*@ \rep @*)#INS#}(*@ \rep @*)
\end{lstlisting}
\end{CenteredBox}
& 
\texttt{texte}
 \\ \hline
    % Commande foot ----------------------------------------------------------------------------------------------------------------------------------------------------------------
foot
&
\vspace{2pt}
\begin{CenteredBox}
  \begin{lstlisting}
foot:=\footnote{#INS#(*@ \rep @*)#INS#}(*@ \rep @*)
\end{lstlisting}
\end{CenteredBox}
& 
texte\footnote{Note de bas de page associé au \emph{texte}} 
 \\ \hline

  \end{tabular}
\end{adjustbox}
\end{table}
\end{landscape}

%%%%%%%%%%%%%%%%%%%%% OPERATIONS MATHEMATIQUES %%%%%%%%%%%%%%%%%%%%%%%%%%%%%%%%%%%%%%%%
\begin{landscape}
\small  % Switch from 12pt to 11pt; otherwise, table won't fit
\setlength\LTleft{0pt}            % default: \parindent
\setlength\LTright{0pt}           % default: \fill
\begin{table}[t]
\caption{Opérations mathématiques}
\begin{adjustbox}{width=\linewidth}
\begin{tabular}{|x{2.6cm}|y{15cm}|x{5cm}|}
\hline 
Commande complétion &    
 Macro correspondante & Résultat produit \\
\hline \hline
   % Commande frac ----------------------------------------------------------------------------------------------------------------------------------------------------------------
fr
&
\vspace{2pt}
\begin{CenteredBox}
  \begin{lstlisting}
fr:=\frac{#INS#(*@ \rep @*)#INS#}{(*@ \rep @*)}(*@ \rep @*)
\end{lstlisting}
\end{CenteredBox}
& 
$\frac{1}{2}$
 \\ \hline
  % Commande sqrt ----------------------------------------------------------------------------------------------------------------------------------------------------------------
sq
&
\vspace{2pt}
\begin{CenteredBox}
  \begin{lstlisting}
sq:=\sqrt{#INS#(*@ \rep @*)#INS#}(*@ \rep @*)
\end{lstlisting}
\end{CenteredBox}
& 
$\sqrt{2}$
 \\ \hline
   % Commande inner prod ----------------------------------------------------------------------------------------------------------------------------------------------------------------
ip
&
\vspace{2pt}
\begin{CenteredBox}
  \begin{lstlisting}
ip:=\langle #INS#(*@ \rep @*)#INS#, (*@ \rep @*) \rangle (*@ \rep @*)
\end{lstlisting}
\end{CenteredBox}
& 
$\langle \mathbf{x},\mathbf{y}\rangle$
 \\ \hline
    % Commande sum ----------------------------------------------------------------------------------------------------------------------------------------------------------------
sum
&
\vspace{2pt}
\begin{CenteredBox}
  \begin{lstlisting}
sum:=\sum_{#INS#(*@ \rep @*)#INS#}^{(*@ \rep @*)}(*@ \rep @*)
\end{lstlisting}
\end{CenteredBox}
& 
$\displaystyle \sum_{n=1}^{N}\frac 1 n$
 \\ \hline
    % Commande int ----------------------------------------------------------------------------------------------------------------------------------------------------------------
int
&
\vspace{2pt}
\begin{CenteredBox}
  \begin{lstlisting}
int:=\int_{#INS#(*@ \rep @*)#INS#}^{(*@ \rep @*)}(*@ \rep @*)
\end{lstlisting}
\end{CenteredBox}
& 
$\displaystyle \int_{a}^{b}\frac 1 x\,\mathrm{d}x$
 \\ \hline
     % Commande prod ----------------------------------------------------------------------------------------------------------------------------------------------------------------
prod
&
\vspace{2pt}
\begin{CenteredBox}
  \begin{lstlisting}
prod:=\prod_{#INS#(*@ \rep @*)#INS#}^{(*@ \rep @*)}(*@ \rep @*)
\end{lstlisting}
\end{CenteredBox}
& 
$\displaystyle \prod_{n=1}^{N}x_n$
 \\ \hline
    % Commande bcap ----------------------------------------------------------------------------------------------------------------------------------------------------------------
bcap
&
\vspace{2pt}
\begin{CenteredBox}
  \begin{lstlisting}
bcap:=\bcap_{#INS#(*@ \rep @*)#INS#}^{(*@ \rep @*)}(*@ \rep @*)
\end{lstlisting}
\end{CenteredBox}
& 
$\displaystyle \bigcap_{n=1}^{N}x_n$
 \\ \hline
    % Commande lim ----------------------------------------------------------------------------------------------------------------------------------------------------------------
lim
&
\vspace{2pt}
\begin{CenteredBox}
  \begin{lstlisting}
lim:=\lim_{#INS#(*@ \rep @*)#INS#}(*@ \rep @*)
\end{lstlisting}
\end{CenteredBox}
& 
$\displaystyle \lim_{x\to +\infty} \mathrm{e}^{x}$
 \\ \hline
    % Commande hat ----------------------------------------------------------------------------------------------------------------------------------------------------------------
ht
&
\vspace{2pt}
\begin{CenteredBox}
  \begin{lstlisting}
ht:=\hat{#INS#(*@ \rep @*)#INS#}(*@ \rep @*)
\end{lstlisting}
\end{CenteredBox}
& 
$\hat x$
 \\ \hline
   % Commande widehat ----------------------------------------------------------------------------------------------------------------------------------------------------------------
wht
&
\vspace{2pt}
\begin{CenteredBox}
  \begin{lstlisting}
wht:=\widehat{#INS#(*@ \rep @*)#INS#}(*@ \rep @*)
\end{lstlisting}
\end{CenteredBox}
& 
$\widehat x$
 \\ \hline
   % Commande underline ----------------------------------------------------------------------------------------------------------------------------------------------------------------
unl
&
\vspace{2pt}
\begin{CenteredBox}
  \begin{lstlisting}
unl:=\underline{#INS#(*@ \rep @*)#INS#}(*@ \rep @*)
\end{lstlisting}
\end{CenteredBox}
& 
$\underline{x}$
 \\ \hline
   % Commande ovl ----------------------------------------------------------------------------------------------------------------------------------------------------------------
ovl
&
\vspace{2pt}
\begin{CenteredBox}
  \begin{lstlisting}
ovl:=\overline{#INS#(*@ \rep @*)#INS#}(*@ \rep @*)
\end{lstlisting}
\end{CenteredBox}
& 
$\overline{x}$
 \\ \hline



 
\end{tabular}
\end{adjustbox}
\end{table}
\end{landscape}
 
 %%%%%%%%%%%%%%%%%%%%% OPERATIONS MATHEMATIQUES %%%%%%%%%%%%%%%%%%%%%%%%%%%%%%%%%%%%%%%%
\begin{landscape}
\small  % Switch from 12pt to 11pt; otherwise, table won't fit
\setlength\LTleft{0pt}            % default: \parindent
\setlength\LTright{0pt}           % default: \fill
\begin{table}[t]
\caption{Opérations mathématiques}
\begin{adjustbox}{width=\linewidth}
\begin{tabular}{|x{2.6cm}|y{15cm}|x{5cm}|}
\hline 
Commande complétion &    
 Macro correspondante & Résultat produit \\
\hline \hline
  % Commande vect ----------------------------------------------------------------------------------------------------------------------------------------------------------------
vc
&
\vspace{2pt}
\begin{CenteredBox}
  \begin{lstlisting}
vc:=\vec{#INS#(*@ \rep @*)#INS#}(*@ \rep @*)
\end{lstlisting}
\end{CenteredBox}
& 
$\vec x$
 \\ \hline
   % Commande long vect ----------------------------------------------------------------------------------------------------------------------------------------------------------------
lvc
&
\vspace{2pt}
\begin{CenteredBox}
  \begin{lstlisting}
lvc:=\overrightarrow{#INS#(*@ \rep @*)#INS#}(*@ \rep @*)
\end{lstlisting}
\end{CenteredBox}
& 
$\overrightarrow{x}$
 \\ \hline
    % Commande xrightarrow ----------------------------------------------------------------------------------------------------------------------------------------------------------------
xra
&
\vspace{2pt}
\begin{CenteredBox}
  \begin{lstlisting}
xra:=\xrightarrow{#INS#(*@ \rep @*)#INS#}(*@ \rep @*)
\end{lstlisting}
\end{CenteredBox}
& 
$x \xrightarrow{\text{texte}} y$
 \\ \hline
   % Commande binom ----------------------------------------------------------------------------------------------------------------------------------------------------------------
binom
&
\vspace{2pt}
\begin{CenteredBox}
  \begin{lstlisting}
binom:=\binom{#INS#(*@ \rep @*)#INS#}{(*@ \rep @*)}(*@ \rep @*)
\end{lstlisting}
\end{CenteredBox}
& 
$\binom{n}{k}$
 \\ \hline
   % Commande underbrace----------------------------------------------------------------------------------------------------------------------------------------------------------------
unb
&
\vspace{2pt}
\begin{CenteredBox}
  \begin{lstlisting}
unb:=\underbrace{#INS#(*@ \rep @*)#INS#}_{(*@ \rep @*)}(*@ \rep @*)
\end{lstlisting}
\end{CenteredBox}
& 
$\underbrace{1+\cdots+100}_{\text{somme}}$
 \\ \hline
    % Commande underbrace----------------------------------------------------------------------------------------------------------------------------------------------------------------
ovb
&
\vspace{2pt}
\begin{CenteredBox}
  \begin{lstlisting}
ovb:=\overbrace{#INS#(*@ \rep @*)#INS#}^{(*@ \rep @*)}(*@ \rep @*)
\end{lstlisting}
\end{CenteredBox}
& 
$\overbrace{1+\cdots+100}^{5500}$
 \\ \hline
     % Commande underset----------------------------------------------------------------------------------------------------------------------------------------------------------------
uns
&
\vspace{2pt}
\begin{CenteredBox}
  \begin{lstlisting}
uns:=\underset{#INS#(*@ \rep @*)#INS#}^{(*@ \rep @*)}(*@ \rep @*)
\end{lstlisting}
\end{CenteredBox}
& 
$\underset{?}{=}$
 \\ \hline
     % Commande overset----------------------------------------------------------------------------------------------------------------------------------------------------------------
ovs
&
\vspace{2pt}
\begin{CenteredBox}
  \begin{lstlisting}
ovs:=\overset{#INS#(*@ \rep @*)#INS#}^{(*@ \rep @*)}(*@ \rep @*)
\end{lstlisting}
\end{CenteredBox}
& 
$\overset{?}{=}$
 \\ \hline
    % Commande mathbb ----------------------------------------------------------------------------------------------------------------------------------------------------------------
mbb
&
\vspace{2pt}
\begin{CenteredBox}
  \begin{lstlisting}
mbb:=\mathbf{#INS#(*@ \rep @*)#INS#}(*@ \rep @*)
\end{lstlisting}
\end{CenteredBox}
& 
$\mathbb{R}$
 \\ \hline
    % Commande mathcal ----------------------------------------------------------------------------------------------------------------------------------------------------------------
mcal
&
\vspace{2pt}
\begin{CenteredBox}
  \begin{lstlisting}
mcal:=\mathcal{#INS#(*@ \rep @*)#INS#}(*@ \rep @*)
\end{lstlisting}
\end{CenteredBox}
& 
$\mathcal{R}$
 \\ \hline
    % Commande mathbf ----------------------------------------------------------------------------------------------------------------------------------------------------------------
mbf
&
\vspace{2pt}
\begin{CenteredBox}
  \begin{lstlisting}
mbf:=\mathbf{#INS#(*@ \rep @*)#INS#}(*@ \rep @*)
\end{lstlisting}
\end{CenteredBox}
& 
$\mathbf{R}$
 \\ \hline
   % Commande mathrm ----------------------------------------------------------------------------------------------------------------------------------------------------------------
mrm
&
\vspace{2pt}
\begin{CenteredBox}
  \begin{lstlisting}
mrm:=\mathrm{#INS#(*@ \rep @*)#INS#}(*@ \rep @*)
\end{lstlisting}
\end{CenteredBox}
& 
$\mathrm{R}$
 \\ \hline


  \end{tabular}
\end{adjustbox}
\end{table}
\end{landscape}

%%%%%%%%%%%%%%%%%%%%% SYMBOLES MATHEMATIQUES %%%%%%%%%%%%%%%%%%%%%%%%%%%%%%%%%%%%%%%%
\begin{landscape}
\small  % Switch from 12pt to 11pt; otherwise, table won't fit
\setlength\LTleft{0pt}            % default: \parindent
\setlength\LTright{0pt}           % default: \fill
\begin{table}[t]
\caption{Symboles mathématiques}
\begin{adjustbox}{width=\linewidth}
\begin{tabular}{|x{2.6cm}|y{15cm}|x{5cm}|}
\hline 
Commande complétion &    
 Macro correspondante & Résultat produit \\
\hline \hline
  % Commande alpha ----------------------------------------------------------------------------------------------------------------------------------------------------------------
xa
&
\vspace{2pt}
\begin{CenteredBox}
  \begin{lstlisting}
xa:=\alpha
\end{lstlisting}
\end{CenteredBox}
& 
$\alpha$, $A$
 \\ \hline
   % Commande beta ----------------------------------------------------------------------------------------------------------------------------------------------------------------
xb
&
\vspace{2pt}
\begin{CenteredBox}
  \begin{lstlisting}
xb:=\beta
\end{lstlisting}
\end{CenteredBox}
& 
$\beta$, $B$
 \\ \hline
  % Commande gamma ----------------------------------------------------------------------------------------------------------------------------------------------------------------
xg, xdg, xcg
&
\vspace{2pt}
\begin{CenteredBox}
  \begin{lstlisting}
xg:=\gamma, xdg:=\digamma, xcg:=\Gamma
\end{lstlisting}
\end{CenteredBox}
& 
$\gamma$, $\digamma$, $\Gamma$
 \\ \hline
 % Commande delta ----------------------------------------------------------------------------------------------------------------------------------------------------------------
xd, xcd
&
\vspace{2pt}
\begin{CenteredBox}
  \begin{lstlisting}
xd:=\delta, xcd:=\Delta
\end{lstlisting}
\end{CenteredBox}
& 
$\delta$, $\Delta$
 \\ \hline
 % Commande epsilon ----------------------------------------------------------------------------------------------------------------------------------------------------------------
xe, xve
&
\vspace{2pt}
\begin{CenteredBox}
  \begin{lstlisting}
xe:=\epsilon, xve:=\varepsilon
\end{lstlisting}
\end{CenteredBox}
& 
$\epsilon$, $\varepsilon$, $E$
 \\ \hline
 % Commande zeta ----------------------------------------------------------------------------------------------------------------------------------------------------------------
xz
&
\vspace{2pt}
\begin{CenteredBox}
  \begin{lstlisting}
xz:=\zeta
\end{lstlisting}
\end{CenteredBox}
& 
$\zeta$, $Z$
 \\ \hline
 % Commande eta ----------------------------------------------------------------------------------------------------------------------------------------------------------------
xet
&
\vspace{2pt}
\begin{CenteredBox}
  \begin{lstlisting}
xet:=\eta
\end{lstlisting}
\end{CenteredBox}
& 
$\eta$, $H$
 \\ \hline
 % Commande theta ----------------------------------------------------------------------------------------------------------------------------------------------------------------
xth, xvth, xcth
&
\vspace{2pt}
\begin{CenteredBox}
  \begin{lstlisting}
xth:=\theta, xvth:=\vartheta, xcth:=\Theta
\end{lstlisting}
\end{CenteredBox}
& 
$\theta$,  $\vartheta$, $\Theta$
 \\ \hline
 % Commande iota ----------------------------------------------------------------------------------------------------------------------------------------------------------------
xi
&
\vspace{2pt}
\begin{CenteredBox}
  \begin{lstlisting}
xi:=\iota
\end{lstlisting}
\end{CenteredBox}
& 
$\iota$, $I$
 \\ \hline
 % Commande kappa ----------------------------------------------------------------------------------------------------------------------------------------------------------------
xk, xvk
&
\vspace{2pt}
\begin{CenteredBox}
  \begin{lstlisting}
xk:=\kappa, xvk:=\varkappa
\end{lstlisting}
\end{CenteredBox}
& 
$\kappa$, $\varkappa$, $K$
 \\ \hline
  % Commande lambda ----------------------------------------------------------------------------------------------------------------------------------------------------------------
xl
&
\vspace{2pt}
\begin{CenteredBox}
  \begin{lstlisting}
xl:=\lambda, xcl:=\Lambda
\end{lstlisting}
\end{CenteredBox}
& 
$\lambda$, $\Lambda$
 \\ \hline
  % Commande mu ----------------------------------------------------------------------------------------------------------------------------------------------------------------
xm
&
\vspace{2pt}
\begin{CenteredBox}
  \begin{lstlisting}
xm:=\mu
\end{lstlisting}
\end{CenteredBox}
& 
$\mu$, $M$
 \\ \hline

  \end{tabular}
\end{adjustbox}
\end{table}
\end{landscape}

%%%%%%%%%%%%%%%%%%%%% SYMBOLES MATHEMATIQUES (SUITE) %%%%%%%%%%%%%%%%%%%%%%%%%%%%%%%%%%%%%%%%
\begin{landscape}
\small  % Switch from 12pt to 11pt; otherwise, table won't fit
\setlength\LTleft{0pt}            % default: \parindent
\setlength\LTright{0pt}           % default: \fill
\begin{table}[t]
\caption{Symboles mathématiques}
\begin{adjustbox}{width=\linewidth}
\begin{tabular}{|x{2.6cm}|y{15cm}|x{5cm}|}
\hline 
Commande complétion &    
 Macro correspondante & Résultat produit \\
\hline \hline
  % Commande nu ----------------------------------------------------------------------------------------------------------------------------------------------------------------
xn
&
\vspace{2pt}
\begin{CenteredBox}
  \begin{lstlisting}
xn:=\nu
\end{lstlisting}
\end{CenteredBox}
& 
$\nu$, $N$
 \\ \hline
  % Commande xi ----------------------------------------------------------------------------------------------------------------------------------------------------------------
xx, xcx
&
\vspace{2pt}
\begin{CenteredBox}
  \begin{lstlisting}
xx:=\xi, xcx:=\Xi
\end{lstlisting}
\end{CenteredBox}
& 
$\xi$, $\Xi$
 \\ \hline
  % Commande pi ----------------------------------------------------------------------------------------------------------------------------------------------------------------
xp, xvp, xcp
&
\vspace{2pt}
\begin{CenteredBox}
  \begin{lstlisting}
xp:=\pi, xvp:=\varpi, xcp:=\Pi
\end{lstlisting}
\end{CenteredBox}
& 
$\pi$, $\varpi$, $\Pi$
 \\ \hline
  % Commande rho ----------------------------------------------------------------------------------------------------------------------------------------------------------------
xr, xvr
&
\vspace{2pt}
\begin{CenteredBox}
  \begin{lstlisting}
xr:=\rho, xvr:=\varrho
\end{lstlisting}
\end{CenteredBox}
& 
$\rho$, $\varrho$, $P$
 \\ \hline
  % Commande sigma ----------------------------------------------------------------------------------------------------------------------------------------------------------------
xs, xcs
&
\vspace{2pt}
\begin{CenteredBox}
  \begin{lstlisting}
xs:=\sigma, xcs:=\Sigma
\end{lstlisting}
\end{CenteredBox}
& 
$\sigma$, $\Sigma$
 \\ \hline
  % Commande tau ----------------------------------------------------------------------------------------------------------------------------------------------------------------
xt
&
\vspace{2pt}
\begin{CenteredBox}
  \begin{lstlisting}
xt:=\tau
\end{lstlisting}
\end{CenteredBox}
& 
$\tau$, $T$
 \\ \hline
  % Commande upsilon ----------------------------------------------------------------------------------------------------------------------------------------------------------------
xu, xcu
&
\vspace{2pt}
\begin{CenteredBox}
  \begin{lstlisting}
xu:=\upsilon, xcu:=\Upsilon
\end{lstlisting}
\end{CenteredBox}
& 
$\upsilon$, $\Upsilon$
 \\ \hline
  % Commande phi ----------------------------------------------------------------------------------------------------------------------------------------------------------------
xph, xvph, xcph
&
\vspace{2pt}
\begin{CenteredBox}
  \begin{lstlisting}
xph:=\phi, xvph:=\varphi, xcph:=\Phi
\end{lstlisting}
\end{CenteredBox}
& 
$\phi$, $\varphi$, $\Phi$
 \\ \hline
  % Commande chi ----------------------------------------------------------------------------------------------------------------------------------------------------------------
xch
&
\vspace{2pt}
\begin{CenteredBox}
  \begin{lstlisting}
xch:=\chi
\end{lstlisting}
\end{CenteredBox}
& 
$\chi$, $X$
 \\ \hline
  % Commande psi ----------------------------------------------------------------------------------------------------------------------------------------------------------------
xps, xcps
&
\vspace{2pt}
\begin{CenteredBox}
  \begin{lstlisting}
xps:=\psi, xcps:=\Psi
\end{lstlisting}
\end{CenteredBox}
& 
$\psi$, $\Psi$
 \\ \hline
  % Commande omega ----------------------------------------------------------------------------------------------------------------------------------------------------------------
xo, xco
&
\vspace{2pt}
\begin{CenteredBox}
  \begin{lstlisting}
xo:=\omega, xco:=\Omega
\end{lstlisting}
\end{CenteredBox}
& 
$\omega$, $\Omega$
 \\ \hline
 
 \end{tabular}
\end{adjustbox}
\end{table}
\end{landscape}

\end{document}  